
\usepackage{acronym}

% === Acronyms, alphabetical ===
\acrodef{ADC}{analogue to digital converter}
\acrodef{AE}{Acoustic Emission}
\acrodef{AIC}{Akaike information criteria}
\acrodef{API}{application programming interface}
\acrodef{AWGN}{Additive White Gaussian Noise}
\acrodef{CDF}{cumulative distribution function}
\acrodef{CRLB}{Cram\'{e}r-Rao lower bound}
\acrodef{DFT}{Discrete Fourier Transform}
\acrodef{DMA}{direct memory access}
\acrodef{DUT}{device under test}
\acrodef{ECE}{Electrical and Computer Engineering}
\acrodef{EM}{electromagnetic}
\acrodef{FFT}{fast Fourier Transform}
\acrodef{FIR}{Finite Impulse Response}
\acrodef{FTSP}{flooding time synchronisation protocol}
\acrodef{GF}{growth and form}
\acrodef{GPS}{global positioning system}
\acrodef{GUI}{graphical user interface}
\acrodef{IR}{infra-red}
\acrodef{IRQ}{Interrupt Request}
\acrodef{LiPo}{Lithium Polymer}
\acrodef{LED}{light emitting diode}
\acrodef{LPF}{lowpass filter}
\acrodef{LTI}{linear, time-invariant}
\acrodef{mbps}{megabit per second}
\acrodef{MCU}{microcontroller unit}
\acrodef{MFA}{Microfibril angle}
\acrodef{ML}{maximum likelihood}
\acrodef{MoE}{modulus of elasticity}
\acrodef{MoR}{modulus of rupture}
\acrodef{MVA}{Multivariate analysis}
\acrodef{NIR}{Near-infrared}
\acrodef{NZSoF}{New Zealand School of Forestry}
\acrodef{op-amp}{operational amplifier}
\acrodef{PDA}{personal digital assistant}
\acrodef{PCB}{printed circuit board}
\acrodef{PDF}{probability distribution function}
\acrodef{PDS}{passive data structure} 
\acrodef{PGA}{programmable gain amplifier}
\acrodef{ppm}{parts per million}
\acrodef{PPS}{Pulse Per Second}
\acrodef{PSD}{Power Spectral Density}
\acrodef{RMS}{root mean square}
\acrodef{RV}{Random Variable}
\acrodef{SLS}{standard linear solid}
\acrodef{SNR}{Signal to Noise Ratio}
\acrodef{SSC}{synchronous serial controller}
\acrodef{TC}{timer-counter}
\acrodef{TCXO}{temperature-controlled crystal oscillator}
\acrodef{ToF}{time of flight}
\acrodef{TTF}{Time to Fix}
\acrodef{UC}{University of Canterbury}
\acrodef{USB}{universal serial bus}
\acrodef{VSD}{Voltage spectral density}
\acrodef{WSN}{wireless sensor network}
\acrodef{WTT}{Wireless Treetap}

\newcommand{\defsymg}[3]{
	\newglossaryentry{#1}{
		name={\ensuremath{#2}},
		description={#3},
		sort={#1}
	}
	\csdef{#1}{\gls{#1}}
}

% === Text conveniences
\newcommand{\lame}{Lam\'{e}}

% === Development options ===
\newcommand{\mcomm}[1]{\emph{(#1)}}				%comment in brackets
\newcommand{\needcite}{\emph{(NEED CITE)}}

% === Mathematical expressions ===
\newcommand{\encp}[1]{\left(#1\right)}			%enclose in brackets
\newcommand{\encs}[1]{\left[#1\right]}			%enclose in square brackets
\newcommand{\encc}[1]{\left\{#1\right\}}			%enclose in curly braces
\newcommand{\expp}[1]{\exp{\encp{#1}}}
\newcommand{\maxval}[1]{\mathrm{max}\encp{#1}}
\newcommand{\real}[1]{\operatorname{Re}\encc{#1}}
\newcommand{\imag}[1]{\operatorname{Im}\encc{#1}}
\newcommand{\conj}[1]{#1^{\ast}}
\newcommand{\spunit}{\,}						%spacing between value and its unit
\newcommand{\scinote}[1]{\times 10^{#1}}
\newcommand{\unit}[1]{\spunit \mathrm{#1}}
\newcommand{\tun}[1]{\spunit#1}

%%% NOTE: commath defines the \pd macro for partial derivatives
%%%	TODO: eliminate these derivate macros in favour of commath.
\newcommand{\dx}[1]{\mathrm{d} #1}
\newcommand{\dpndx}[2]{\frac{\partial^{#1}}{\partial #2^{#1}}}	%partial nth derivative
\newcommand{\dpdx}[1]{\dpndx{}{#1}}
\newcommand{\dpddx}[1]{\dpndx{2}{#1}}							%partial second derivative

\DeclareMathOperator{\Tr}{Tr}
\newcommand{\unitv}[1]{\boldsymbol{\hat{#1}}}
\newcommand{\axisv}[1]{\boldsymbol{\hat{e}_{#1}}}
\newcommand{\diverg}[1]{\nabla \cdot #1}
\newcommand{\symdiv}{\nabla_{\mathrm{s}}}						%symmetric vector derivative

% === Figures ===
\newcommand{\sref}[1]{(\subref{#1})}
\newcommand{\pdffig}[2][0.8]{\includegraphics[width=#1\textwidth]{#2}}

% === Code ===
\newcommand{\codeinline}[1]{\texttt{#1}}

% === References ===
\newcommand{\reffig}[1]{Figure \ref{#1}}
\newcommand{\refsfig}[2]{\reffig{#1} \sref{#2}}
\newcommand{\refsec}[1]{Section \ref{#1}}
\newcommand{\reftab}[1]{Table \ref{#1}}
\newcommand{\refchap}[1]{Chapter \ref{#1}}
\newcommand{\refapp}[1]{Appendix \ref{#1}}
\newcommand{\reflst}[1]{Listing \ref{#1}}
%\newcommand{\refeq}[1]{(\ref{#1})}



\usepackage{acronym}

% === Acronyms, alphabetical ===
\acrodef{ADC}{Analogue to Digital Converter}
\acrodef{AIC}{Akaike information criteria}
\acrodef{API}{Application Programming Interface}
\acrodef{DFT}{Discrete Fourier Transform}
\acrodef{ECE}{Electrical and Computer Engineering}
\acrodef{EM}{electromagnetic}
\acrodef{FFT}{fast Fourier Transform}
\acrodef{FTSP}{Flooding Time Synchronisation Protocol}
\acrodef{GF}{growth and form}
\acrodef{GPS}{Global Positioning System}
\acrodef{GUI}{Graphical User Interface}
\acrodef{IRQ}{Interrupt Request}
\acrodef{mbps}{megabit per second}
\acrodef{MCU}{microcontroller unit}
\acrodef{MFA}{Microfibril angle}
\acrodef{MoE}{Modulus of Elasticity}
\acrodef{MVA}{Multivariate analysis}
\acrodef{NIR}{Near-infrared}
\acrodef{NZSoF}{New Zealand School of Forestry}
\acrodef{PCB}{printed circuit board}
\acrodef{PDS}{passive data structure} 
\acrodef{PGA}{Programmable Gain Amplifier}
\acrodef{ppm}{parts per million}
\acrodef{PPS}{Pulse Per Second}
\acrodef{PSD}{Power Spectral Density}
\acrodef{RMS}{Root Mean Square}
\acrodef{SNR}{Signal to Noise Ratio}
\acrodef{TC}{timer-counter}
\acrodef{TCXO}{temperature-controlled crystal oscillator}
\acrodef{ToF}{Time of flight}
\acrodef{TTF}{Time to Fix}
\acrodef{UC}{University of Canterbury}
\acrodef{USB}{Universal Serial Bus}
\acrodef{VSD}{Voltage spectral density}
\acrodef{WSN}{Wireless sensor network}
\acrodef{WTT}{Wireless Treetap}

% === Development options ===
\newcommand{\mcomm}[1]{\emph{(#1)}}				%comment in brackets
\newcommand{\needcite}{\emph{(NEED CITE)}}

% === Mathematical expressions ===
\DeclarePairedDelimiter{\encp}{(}{)}
\newcommand{\encs}[1]{\left[#1\right]}			%enclose in square brackets
\newcommand{\expp}[1]{\exp{\encp{#1}}}
\newcommand{\real}[1]{\operatorname{Re}(#1)}
\newcommand{\imag}[1]{\operatorname{Im}(#1)}
\newcommand{\spunit}{\,}						%spacing between value and it's unit

%%% NOTE: commath defines the \pd macro for partial derivatives
%%%	TODO: eliminate these derivate macros in favour of commath.
\newcommand{\dx}[1]{\mathrm{d} #1}
\newcommand{\dpndx}[2]{\frac{\partial^{#1}}{\partial #2^{#1}}}	%partial nth derivative
\newcommand{\dpdx}[1]{\dpndx{}{#1}}
\newcommand{\dpddx}[1]{\dpndx{2}{#1}}							%partial second derivative

\DeclareMathOperator{\Tr}{Tr}
\newcommand{\unitv}[1]{\boldsymbol{\hat{e_{#1}}}}
\newcommand{\diverg}[1]{\nabla \cdot #1}

% === Figures ===
\newcommand{\sref}[1]{(\subref{#1})}
\newcommand{\pdffig}[2][0.8]{\includegraphics[width=#1\textwidth]{#2}}

% === Code ===
\newcommand{\codeinline}[1]{\texttt{#1}}

% === References ===
\newcommand{\reffig}[1]{Figure \ref{#1}}
\newcommand{\refsfig}[2]{\reffig{#1} \sref{#2}}
\newcommand{\refsec}[1]{Section \ref{#1}}
\newcommand{\reftab}[1]{Table \ref{#1}}
\newcommand{\refchap}[1]{Chapter \ref{#1}}
\newcommand{\refappendix}[1]{Appendix \ref{#1}}
%\newcommand{\refeq}[1]{(\ref{#1})}

